\chapter{White Space}

  White space is key to readable code.
  As such it is key to develop a singular style for it's use across a code base.

  \section{Indentation}

    \begin{keybox}[Spaces]
      \textbf{Use spaces only for indentation.}
    \end{keybox}

    Each indentation level should correspond to \textbf{2} spaces.

    All modern editors can be set to emit spaces when the tab key is hit.

  \section{Newlines}

    Newlines should be used to break code into logical segments.

    However, minimize the use of newlines in code.
    This is more guidance than strict rule.
    In particular, don't put more than one or two blank lines between functions, resist starting functions with a blank line, don't end functions with a blank line, and be sparing with your use of blank lines.

    The basic principle is: The more code that fits on one screen, the easier it is to follow and understand the control flow of the program.
    Use whitespace purposefully to provide separation in that flow.

  \section{Line Length}

    Each line of text in the code should be at most 100 characters in length (including indentation).

    A line may exceed 100 characters if one of the following is met:
    \begin{itemize}
      \item It is a comment which is not able to be split without hindering readability or usability (e.g. ease of copy-paste or auto linking)
      \item A String literal which cannot be easily wrapped
      \item An include statement
      \item A header guard
      \item A using declaration
    \end{itemize}