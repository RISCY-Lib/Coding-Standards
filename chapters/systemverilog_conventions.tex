% cspell:ignore POSIX

\chapter{SystemVerilog Conventions}
\label{systemverilog_conventions}

  This chapter seeks to standardize aesthetic aspects of SystemVerilog.

  \section{General File Rules}
  \label{systemverilog_conventions:general_file_rules}

    \subsection{File Extensions}
    \label{systemverilog_conventions:general_file_rules:file_extensions}
      Use the \texttt{.sv} extension for SystemVerilog files (or \texttt{.svh} for files that are included via the preprocessor).

      File extensions have the following meanings:

      \begin{itemize}
      \item \texttt{.sv} indicates a SystemVerilog file.
      \item \texttt{.svh} indicates a SystemVerilog header file intended to be included in another file using the \texttt{`{}include} directive.
      \item \texttt{.v} indicates a Verilog file.
      \item \texttt{.vh} indicates a Verilog header file.
      \end{itemize}

      Only \texttt{.sv} and \texttt{.v} files are intended to be compilation units.
      \texttt{.svh} and \texttt{.vh} files may only be \texttt{`{}include}d into other files.

      With exceptions of netlist files, each \texttt{.sv} or \texttt{.v} file should contain only one module, and the name should be associated.
      For instance, file \texttt{foo.sv} should contain only the module \texttt{foo}.

    \subsection{Characters}
    \label{systemverilog_conventions:general_file_rules:characters}

      \begin{keybox}
        Use only ASCII characters with UNIX-style line endings(\texttt{\" \textbackslash n\"}).
      \end{keybox}

    \subsection{POSIX File Endings}
    \label{systemverilog_conventions:general_file_rules:posix_file_endings}

      \begin{keybox}
        All lines on non-empty files must end with a newline (\texttt{\" \textbackslash n\"}).
      \end{keybox}

    \subsection{Line Length}
    \label{systemverilog_conventions:general_file_rules:systemverilog_conventions:general_file_rules:line_length}

      \begin{keybox}
        Wrap the code at 100 characters per line.
      \end{keybox}

      The maximum line length for style-compliant Verilog code is 100 characters per line.

      Exceptions:

      \begin{itemize}
        \item Any place where line wraps are impossible (for example, an include path might extend past 100 characters).
      \end{itemize}

      Line-wrapping contains additional guidelines on how to wrap long lines.

    \subsection{No Tabs}
    \label{systemverilog_conventions:general_file_rules:no_tabs}

      \begin{keybox}
        Do not use tabs anywhere.
      \end{keybox}

      Use spaces to indent or align text. See Indentation for rules about indentation and wrapping.

      To convert tabs to spaces on any file, you can use the UNIX expand utility.

    \subsection{No Trailing Spaces}
    \label{systemverilog_conventions:general_file_rules:no_trailing_spaces}

      \begin{keybox}
        Delete trailing whitespace at the end of lines.
      \end{keybox}

  \section{Begin and End}
  \label{systemverilog_conventions:begin_and_end}

    \begin{keybox}
      Use \lstinline{begin} and \lstinline{end} unless the entire statement fits on one line.
    \end{keybox}

    If a statement cannot fit on a single line then the statement(s) must be enclosed in a \lstinline{begin} and \lstinline{end} block.

    \begin{goodbox}
      \lstinputlisting{./conventions/multiline_begin_end.sv}
    \end{goodbox}

    \begin{goodbox}
      \lstinputlisting{./conventions/single_line_no_begin_end.sv}
    \end{goodbox}

    \begin{badbox}
      \lstinputlisting{./conventions/multiline_begin_end_bad.sv}
    \end{badbox}

    The \lstinline{begin} must be on the same line as the prior keyword.
    \lstinline{begin} must end the line.
    \lstinline{end} must occupy it's own line.
    When multiple \lstinline{end} are found sequentially, they should be accommodated by a label (or comment) to clarify which \lstinline{begin} this \lstinline{end} corresponds with.

    \begin{goodbox}
      \lstinputlisting{./conventions/begin_end_else.sv}
    \end{goodbox}

    \begin{badbox}
      \lstinputlisting{./conventions/begin_end_else_bad.sv}
    \end{badbox}

    The above \lstinline{begin} this \lstinline{end} principles also apply to case statements.

    \begin{goodbox}
      \lstinputlisting{./conventions/case_begin_end.sv}
    \end{goodbox}

    \begin{badbox}
      \lstinputlisting{./conventions/case_begin_end_bad.sv}
    \end{badbox}

  \section{Indentation}
  \label{systemverilog_conventions:indentation}

    \begin{keybox}
      Indentation must be two spaces per level.
      Tab characters must be expanded to spaces.
    \end{keybox}

    Modern editors can/should be set to produce spaces when the tab key is hit.

    \begin{keybox}
      An additional level of indentation must be added between paired keywords.
      (e.g. \lstinline{begin / end}, \lstinline{module / endmodule}, \lstinline{task / endtask}, \lstinline{function / endfunction}, etc.)
    \end{keybox}

    \subsection{Wrapping Lines}
    \label{systemverilog_conventions:indentation:wrapping_lines}

      When wrapping long expressions to multiple lines one of the following approaches should be used:

      \begin{itemize}
        \item Indent the continuation line by 4-spaces.
        \item Align the line continuation with an opening parentheses or brace on the first line.
      \end{itemize}

      Additionally, opening parentheses or braces which end a line in the wrapped expression should be closed with the respective character on their own line.

      \begin{goodbox}
        \lstinputlisting{./conventions/line_wrapping.sv}
      \end{goodbox}

    \subsection{Justification}
    \label{systemverilog_conventions:indentation:justification}

      Virtually every coding standard in the world has moved to using spaces instead of tabs.
      This has mainly been driven by a desire for consistency.

      While tabs allow individual developers to choose their personal preference for spacing of tabs,
        this invariably impacts the spacing of wrapped lines.
      Additionally, when adding tabbed lines to a code base which uses spaces, it is likely that the number of tabs
        will be chosen to match the current code base, rather than the match the true indentation level.

      To avoid these issues spaces are required.

      So far as the number of spaces is concerned this is more personal preference.
      Python's official guidelines specify 4-spaces, while most of Google's coding standards enforce 2-spaces (the Linux kernel guidelines even specify 8!).
      In this document 2 spaces have been chosen because they appear to be the generally more readable method to the authors of these guidelines.

  \section{Spacing}
  \label{systemverilog_conventions:spacing}

    \begin{keybox}
      Use spacing which aids readability.
    \end{keybox}

    \subsection{Tabular Alignment}
    \label{systemverilog_conventions:spacing:tabular_alignment}

      Tabular alignment is the practice of aligning specific parts of multiple related lines.
      This allows for ease of reading and comparing multiple related lines.

      \textbf{The use of tabular alignment is strongly encouraged.}

      \begin{goodbox}
        \lstinputlisting{./conventions/tabular_alignment.sv}
      \end{goodbox}

    \subsection{Expressions}
    \label{systemverilog_conventions:spacing:expressions}

      \textbf{Whitespace should be included on both sides of all non-trivial binary operators.}

      \begin{goodbox}
        \lstinputlisting{./conventions/expression_whitespace.sv}
      \end{goodbox}

      \begin{badbox}
        \lstinputlisting{./conventions/expression_whitespace_bad.sv}
      \end{badbox}

    \subsection{Keywords}
    \label{systemverilog_conventions:spacing:keywords}

      \textbf{Whitespace should be included on both sides of SystemVerilog keywords.}

      The following exceptions apply:
      \begin{itemize}
        \item Before a keyword at the beginning of a line
        \item After a keyword at the end of a line
        \item Before keywords which are immediately preceded by an opening parentheses or brace
      \end{itemize}

      % TODO: Code example

    \subsection{Comma Delimited Lists}
    \label{systemverilog_conventions:spacing:comma_delimited_lists}

      \textbf{When multiple items on a line are comma delimited each comma must be followed by a space.}

      % TODO: Code example

    \subsection{Definitions}
    \label{systemverilog_conventions:spacing:definitions}

      \textbf{Add a space around the packed dimensions.}

      Do not add a space -
      \begin{itemize}
        \item Between identified and unpacked dimensions
        \item Between multiple dimensions
      \end{itemize}

      % TODO: Code example

    \subsection{Parameterized Types}
    \label{systemverilog_conventions:spacing:parameterized_types}

      \textbf{Add a single space before type parameters, except when part of a qualified name.}

      % TODO: Code example

    \subsection{Labels and Case Items}
    \label{systemverilog_conventions:spacing:labels_and_case_items}

      \textbf{Do not add a space prior to the colon in a Label or Case item.}
      \textbf{Add at least one space after the colon in a Label or Case item.}

      % TODO: Code example

    \subsection{Calls}
    \label{systemverilog_conventions:spacing:calls}

      \textbf{Functions, Task, and Macro calls must not have any space between the name and the the opening parenthesis.}
      % TODO: Code example

  \section{Parentheses}
  \label{systemverilog_conventions:parentheses}

    \begin{keybox}
      Parentheses should be used to make operations unambiguous.
    \end{keybox}

    In any situation where most people would have to spend some modicum of thought, parentheses should be used instead to make the order of operations clear.

  \section{Ternary Expressions}
  \label{systemverilog_conventions:ternary_expressions}

    \begin{keybox}
      Nested ternary expressions must be enclosed in parentheses unless they are in the false condition.
      \textbf{At most} 3 ternary expressions may be nested together.
    \end{keybox}

    % TODO: Code Example

  \section{Comments}
  \label{systemverilog_conventions:comments}

    \begin{keybox}
      C++ style comments (\lstinline{// comment}) are preferred.
      C style comments (\lstinline{/* comment */} may also be used)
    \end{keybox}

    The following principles should be followed regarding comments:
    \begin{itemize}
      \item A comment on its own line describes the code that immediately follows.
      \item Inline comments describe that line of code
      \item Header style comments (see below) can be used to seperate different functional parts within a module or class.
    \end{itemize}

    \begin{goodbox}
      \lstinputlisting{./conventions/comments.sv}
    \end{goodbox}

  \section{Declarations}
  \label{systemverilog_conventions:declarations}

    \begin{keybox}
      Signals must be declared before use.
      Implicit net declarations \textbf{must} not be used.
    \end{keybox}
